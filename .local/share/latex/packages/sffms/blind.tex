\documentclass[submission]{sffms}
\author{H. G. Wells}
\surname{Wells}
\title{The Country of the Blind}
\runningtitle{Blind}
\wordcount{9600}
% copyright lapsed in the United States of America
\disposable
\begin{document}



Three hundred miles and more from Chimborazo, one hundred from the
snows of Cotopaxi, in the wildest wastes of Ecuador's Andes, there
lies that mysterious mountain valley, cut off from all the world of
men, the Country of the Blind.  Long years ago that valley lay so far
open to the world that men might come at last through frightful gorges
and over an icy pass into its equable meadows, and thither indeed men
came, a family or so of Peruvian half-breeds fleeing from the lust and
tyranny of an evil Spanish ruler. Then came the stupendous outbreak of
Mindobamba, when it was night in Quito for seventeen days, and the
water was boiling at Yaguachi and all the fish floating dying even as
far as Guayaquil; everywhere along the Pacific slopes there were
land-slips and swift thawings and sudden floods, and one whole side of
the old Arauca crest slipped and came down in thunder, and cut off the
Country of the Blind for ever from the exploring feet of men. But one
of these early settlers had chanced to be on the hither side of the
gorges when the world had so terribly shaken itself, and he perforce
had to forget his wife and his child and all the friends and
possessions he had left up there, and start life over again in the
lower world. He started it again but ill, blindness overtook him, and
he died of punishment in the mines; but the story he told begot a
legend that lingers along the length of the Cordilleras of the Andes
to this day.

He told of his reason for venturing back from that fastness, into
which he had first been carried lashed to a llama, beside a vast bale
of gear, when he was a child. The valley, he said, had in it all that
the heart of man could desire---sweet water, pasture, an even climate,
slopes of rich brown soil with tangles of a shrub that bore an
excellent fruit, and on one side great hanging forests of pine that
held the avalanches high. Far overhead, on three sides, vast cliffs of
grey-green rock were capped by cliffs of ice; but the glacier stream
came not to them, but flowed away by the farther slopes, and only now
and then huge ice masses fell on the valley side. In this valley it
neither rained nor snowed, but the abundant springs gave a rich green
pasture, that irrigation would spread over all the valley space. The
settlers did well indeed there.  Their beasts did well and multiplied,
and but one thing marred their happiness.  Yet it was enough to mar it
greatly. A strange disease had come upon them and had made all the
children born to them there---and, indeed, several older children
also---blind. It was to seek some charm or antidote against this plague
of blindness that he had with fatigue and danger and difficulty
returned down the gorge. In those days, in such cases, men did not
think of germs and infections, but of sins, and it seemed to him that
the reason of this affliction must he in the negligence of these
priestless immigrants to set up a shrine so soon as they entered the
valley. He wanted a shrine---a handsome, cheap, effectual shrine---to be
erected in the valley; he wanted relics and such-like potent things of
faith, blessed objects and mysterious medals and prayers. In his
wallet he had a bar of native silver for which he would not account;
he insisted there was none in the valley with something of the
insistence of an inexpert liar. They had all clubbed their money and
ornaments together, having little need for such treasure up there, he
said, to buy them holy help against their ill. I figure this dim-eyed
young mountaineer, sunburnt, gaunt, and anxious, hat brim clutched
feverishly, a man all unused to the ways of the lower world, telling
this story to some keen-eyed, attentive priest before the great
convulsion; I can picture him presently seeking to return with pious
and infallible remedies against that trouble, and the infinite dismay
with which he must have faced the tumbled vastness where the gorge had
once come out. But the rest of his story of mischances is lost to me,
save that I know of his evil death after several years. Poor stray
from that remoteness! The stream that had once made the gorge now
bursts from the mouth of a rocky cave, and the legend his poor,
ill-told story set going developed into the legend of a race of blind
men somewhere "over there" one may still hear to-day.

And amidst the little population of that now isolated and forgotten
valley the disease ran its course. The old became groping, the young
saw but dimly, and the children that were born to them never saw at
all. But life was very easy in that snow-rimmed basin, lost to all the
world, with neither thorns nor briers, with no evil insects nor any
beasts save the gentle breed of llamas they had lugged and thrust and
followed up the beds of the shrunken rivers in the gorges up which
they had come. The seeing had become purblind so gradually that they
scarcely noticed their loss. They guided the sightless youngsters
hither and thither until they knew the whole valley marvellously, and
when at last sight died out among them the race lived on. They had
even time to adapt themselves to the blind control of fire, which they
made carefully in stoves of stone. They were a simple strain of people
at the first, unlettered, only slightly touched with the Spanish
civilisation, but with something of a tradition of the arts of old
Peru and of its lost philosophy. Generation followed generation. They
forgot many things; they devised many things. Their tradition of the
greater world they came from became mythical in colour and
uncertain. In all things save sight they were strong and able, and
presently chance sent one who had an original mind and who could talk
and persuade among them, and then afterwards another. These two
passed, leaving their effects, and the little community grew in
numbers and in understanding, and met and settled social and economic
problems that arose.  Generation followed generation. Generation
followed generation. There came a time when a child was born who was
fifteen generations from that ancestor who went out of the valley with
a bar of silver to seek God's aid, and who never returned. Thereabout
it chanced that a man came into this community from the outer
world. And this is the story of that man.

He was a mountaineer from the country near Quito, a man who had been
down to the sea and had seen the world, a reader of books in an
original way, an acute and enterprising man, and he was taken on by a
party of Englishmen who had come out to Ecuador to climb mountains, to
replace one of their three Swiss guides who had fallen ill. He climbed
here and he climbed there, and then came the attempt on Parascotopetl,
the Matterhorn of the Andes, in which he was lost to the outer
world. The story of that accident has been written a dozen
times. Pointer's narrative is the best. He tells how the little party
worked their difficult and almost vertical way up to the very foot of
the last and greatest precipice, and how they built a night shelter
amidst the snow upon a little shelf of rock, and, with a touch of real
dramatic power, how presently they found Nunez had gone from
them. They shouted, and there was no reply; shouted and whistled, and
for the rest of that night they slept no more.

As the morning broke they saw the traces of his fall. It seems
impossible he could have uttered a sound. He had slipped eastward
towards the unknown side of the mountain; far below he had struck a
steep slope of snow, and ploughed his way down it in the midst of a
snow avalanche. His track went straight to the edge of a frightful
precipice, and beyond that everything was hidden. Far, far below, and
hazy with distance, they could see trees rising out of a narrow,
shut-in valley---the lost Country of the Blind. But they did not know
it was the lost Country of the Blind, nor distinguish it in any way
from any other narrow streak of upland valley. Unnerved by this
disaster, they abandoned their attempt in the afternoon, and Pointer
was called away to the war before he could make another attack. To
this day Parascotopetl lifts an unconquered crest, and Pointer's
shelter crumbles unvisited amidst the snows.

And the man who fell survived.

At the end of the slope he fell a thousand feet, and came down in the
midst of a cloud of snow upon a snow-slope even steeper than the one
above. Down this he was whirled, stunned and insensible, but without a
bone broken in his body; and then at last came to gentler slopes, and
at last rolled out and lay still, buried amidst a softening heap of
the white masses that had accompanied and saved him. He came to
himself with a dim fancy that he was ill in bed; then realized his
position with a mountaineer's intelligence and worked himself loose
and, after a rest or so, out until he saw the stars. He rested flat
upon his chest for a space, wondering where he was and what had
happened to him. He explored his limbs, and discovered that several of
his buttons were gone and his coat turned over his head. His knife had
gone from his pocket and his hat was lost, though he had tied it under
his chin. He recalled that he had been looking for loose stones to
raise his piece of the shelter wall. His ice-axe had disappeared.

He decided he must have fallen, and looked up to see, exaggerated by
the ghastly light of the rising moon, the tremendous flight he had
taken. For a while he lay, gazing blankly at the vast, pale cliff
towering above, rising moment by moment out of a subsiding tide of
darkness. Its phantasmal, mysterious beauty held him for a space, and
then he was seized with a paroxysm of sobbing laughter . . . .

After a great interval of time he became aware that he was near the
lower edge of the snow. Below, down what was now a moon-lit and
practicable slope, he saw the dark and broken appearance of
rock-strewn turf He struggled to his feet, aching in every joint and
limb, got down painfully from the heaped loose snow about him, went
downward until he was on the turf, and there dropped rather than lay
beside a boulder, drank deep from the flask in his inner pocket, and
instantly fell asleep . . . .

He was awakened by the singing of birds in the trees far below.

He sat up and perceived he was on a little alp at the foot of a vast
precipice that sloped only a little in the gully down which he and his
snow had come. Over against him another wall of rock reared itself
against the sky. The gorge between these precipices ran east and west
and was full of the morning sunlight, which lit to the westward the
mass of fallen mountain that closed the descending gorge. Below him it
seemed there was a precipice equally steep, but behind the snow in the
gully he found a sort of chimney-cleft dripping with snow-water, down
which a desperate man might venture. He found it easier than it
seemed, and came at last to another desolate alp, and then after a
rock climb of no particular difficulty, to a steep slope of trees. He
took his bearings and turned his face up the gorge, for he saw it
opened out above upon green meadows, among which he now glimpsed quite
distinctly a cluster of stone huts of unfamiliar fashion. At times his
progress was like clambering along the face of a wall, and after a
time the rising sun ceased to strike along the gorge, the voices of
the singing birds died away, and the air grew cold and dark about him.
But the distant valley with its houses was all the brighter for
that. He came presently to talus, and among the rocks he noted---for he
was an observant man---an unfamiliar fern that seemed to clutch out of
the crevices with intense green hands. He picked a frond or so and
gnawed its stalk, and found it helpful.

About midday he came at last out of the throat of the gorge into the
plain and the sunlight. He was stiff and weary; he sat down in the
shadow of a rock, filled up his flask with water from a spring and
drank it down, and remained for a time, resting before he went on to
the houses.

They were very strange to his eyes, and indeed the whole aspect of
that valley became, as he regarded it, queerer and more
unfamiliar. The greater part of its surface was lush green meadow,
starred with many beautiful flowers, irrigated with extraordinary
care, and bearing evidence of systematic cropping piece by piece. High
up and ringing the valley about was a wall, and what appeared to be a
circumferential water channel, from which the little trickles of water
that fed the meadow plants came, and on the higher slopes above this
flocks of llamas cropped the scanty herbage. Sheds, apparently
shelters or feeding-places for the llamas, stood against the boundary
wall here and there. The irrigation streams ran together into a main
channel down the centre of the valley, and this was enclosed on either
side by a wall breast high. This gave a singularly urban quality to
this secluded place, a quality that was greatly enhanced by the fact
that a number of paths paved with black and white stones, and each
with a curious little kerb at the side, ran hither and thither in an
orderly manner.  The houses of the central village were quite unlike
the casual and higgledy-piggledy agglomeration of the mountain
villages he knew; they stood in a continuous row on either side of a
central street of astonishing cleanness, here and there their
parti-coloured facade was pierced by a door, and not a solitary window
broke their even frontage. They were parti-coloured with extraordinary
irregularity, smeared with a sort of plaster that was sometimes grey,
sometimes drab, sometimes slate-coloured or dark brown; and it was the
sight of this wild plastering first brought the word "blind" into the
thoughts of the explorer. "The good man who did that," he thought,
"must have been as blind as a bat."

He descended a steep place, and so came to the wall and channel that
ran about the valley, near where the latter spouted out its surplus
contents into the deeps of the gorge in a thin and wavering thread of
cascade. He could now see a number of men and women resting on piled
heaps of grass, as if taking a siesta, in the remoter part of the
meadow, and nearer the village a number of recumbent children, and
then nearer at hand three men carrying pails on yokes along a little
path that ran from the encircling wall towards the houses. These
latter were clad in garments of llama cloth and boots and belts of
leather, and they wore caps of cloth with back and ear flaps. They
followed one another in single file, walking slowly and yawning as
they walked, like men who have been up all night. There was something
so reassuringly prosperous and respectable in their bearing that after
a moment's hesitation Nunez stood forward as conspicuously as possible
upon his rock, and gave vent to a mighty shout that echoed round the
valley.

The three men stopped, and moved their heads as though they were
looking about them. They turned their faces this way and that, and
Nunez gesticulated with freedom. But they did not appear to see him
for all his gestures, and after a time, directing themselves towards
the mountains far away to the right, they shouted as if in
answer. Nunez bawled again, and then once more, and as he gestured
ineffectually the word "blind" came up to the top of his
thoughts. "The fools must be blind," he said.

When at last, after much shouting and wrath, Nunez crossed the stream
by a little bridge, came through a gate in the wall, and approached
them, he was sure that they were blind. He was sure that this was the
Country of the Blind of which the legends told. Conviction had sprung
upon him, and a sense of great and rather enviable adventure. The
three stood side by side, not looking at him, but with their ears
directed towards him, judging him by his unfamiliar steps. They stood
close together like men a little afraid, and he could see their
eyelids closed and sunken, as though the very balls beneath had shrunk
away. There was an expression near awe on their faces.

"A man," one said, in hardly recognisable Spanish. "A man it is---a man
or a spirit---coming down from the rocks."

But Nunez advanced with the confident steps of a youth who enters upon
life. All the old stories of the lost valley and the Country of the
Blind had come back to his mind, and through his thoughts ran this old
proverb, as if it were a refrain:---

"In the Country of the Blind the One-Eyed Man is King."

"In the Country of the Blind the One-Eyed Man is King."

And very civilly he gave them greeting. He talked to them and used his
eyes.

"Where does he come from, brother Pedro?" asked one.

"Down out of the rocks."

"Over the mountains I come," said Nunez, "out of the country beyond
there---where men can see. From near Bogota---where there are a hundred
thousands of people, and where the city passes out of sight."

"Sight?" muttered Pedro. "Sight?"

"He comes," said the second blind man, "out of the rocks."

The cloth of their coats, Nunez saw was curious fashioned, each with a
different sort of stitching.

They startled him by a simultaneous movement towards him, each with a
hand outstretched. He stepped back from the advance of these spread
fingers.

"Come hither," said the third blind man, following his motion and
clutching him neatly.

And they held Nunez and felt him over, saying no word further until
they had done so.

"Carefully," he cried, with a finger in his eye, and found they
thought that organ, with its fluttering lids, a queer thing in
him. They went over it again.

"A strange creature, Correa," said the one called Pedro. "Feel the
coarseness of his hair. Like a llama's hair."

"Rough he is as the rocks that begot him," said Correa, investigating
Nunez's unshaven chin with a soft and slightly moist hand. "Perhaps he
will grow finer."

Nunez struggled a little under their examination, but they gripped him
firm.

"Carefully," he said again.

"He speaks," said the third man. "Certainly he is a man."

"Ugh!" said Pedro, at the roughness of his coat.

"And you have come into the world?" asked Pedro.

"OUT of the world. Over mountains and glaciers; right over above
there, half-way to the sun. Out of the great, big world that goes
down, twelve days' journey to the sea."

They scarcely seemed to heed him. "Our fathers have told us men may be
made by the forces of Nature," said Correa. "It is the warmth of
things, and moisture, and rottenness---rottenness."

"Let us lead him to the elders," said Pedro.

"Shout first," said Correa, "lest the children be afraid. This is a
marvellous occasion."

So they shouted, and Pedro went first and took Nunez by the hand to
lead him to the houses.

He drew his hand away. "I can see," he said.

"See?" said Correa.

"Yes; see," said Nunez, turning towards him, and stumbled against
Pedro's pail.

"His senses are still imperfect," said the third blind man. "He
stumbles, and talks unmeaning words. Lead him by the hand."

"As you will," said Nunez, and was led along laughing.

It seemed they knew nothing of sight.

Well, all in good time he would teach them.

He heard people shouting, and saw a number of figures gathering
together in the middle roadway of the village.

He found it tax his nerve and patience more than he had anticipated,
that first encounter with the population of the Country of the
Blind. The place seemed larger as he drew near to it, and the smeared
plasterings queerer, and a crowd of children and men and women (the
women and girls he was pleased to note had, some of them, quite sweet
faces, for all that their eyes were shut and sunken) came about him,
holding on to him, touching him with soft, sensitive hands, smelling
at him, and listening at every word he spoke. Some of the maidens and
children, however, kept aloof as if afraid, and indeed his voice
seemed coarse and rude beside their softer notes. They mobbed him. His
three guides kept close to him with an effect of proprietorship, and
said again and again, "A wild man out of the rocks."

"Bogota," he said. "Bogota. Over the mountain crests."

"A wild man---using wild words," said Pedro. "Did you hear
that---"BOGOTA? His mind has hardly formed yet. He has only the
beginnings of speech."

A little boy nipped his hand. "Bogota!" he said mockingly.

"Aye! A city to your village. I come from the great world---where men
have eyes and see."

"His name's Bogota," they said.

"He stumbled," said Correa---" stumbled twice as we came hither."

"Bring him in to the elders."

And they thrust him suddenly through a doorway into a room as black as
pitch, save at the end there faintly glowed a fire. The crowd closed
in behind him and shut out all but the faintest glimmer of day, and
before he could arrest himself he had fallen headlong over the feet of
a seated man. His arm, outflung, struck the face of someone else as he
went down; he felt the soft impact of features and heard a cry of
anger, and for a moment he struggled against a number of hands that
clutched him. It was a one-sided fight. An inkling of the situation
came to him and he lay quiet.

"I fell down," be said; I couldn't see in this pitchy darkness."

There was a pause as if the unseen persons about him tried to
understand his words. Then the voice of Correa said: "He is but newly
formed. He stumbles as he walks and mingles words that mean nothing
with his speech."

Others also said things about him that he heard or understood
imperfectly.

"May I sit up?" he asked, in a pause. "I will not struggle against you
again."

They consulted and let him rise.

The voice of an older man began to question him, and Nunez found
himself trying to explain the great world out of which he had fallen,
and the sky and mountains and such-like marvels, to these elders who
sat in darkness in the Country of the Blind. And they would believe
and understand nothing whatever that he told them, a thing quite
outside his expectation. They would not even understand many of his
words. For fourteen generations these people had been blind and cut
off from all the seeing world; the names for all the things of sight
had faded and changed; the story of the outer world was faded and
changed to a child's story; and they had ceased to concern themselves
with anything beyond the rocky slopes above their circling wall. Blind
men of genius had arisen among them and questioned the shreds of
belief and tradition they had brought with them from their seeing
days, and had dismissed all these things as idle fancies and replaced
them with new and saner explanations. Much of their imagination had
shrivelled with their eyes, and they had made for themselves new
imaginations with their ever more sensitive ears and
finger-tips. Slowly Nunez realised this: that his expectation of
wonder and reverence at his origin and his gifts was not to be borne
out; and after his poor attempt to explain sight to them had been set
aside as the confused version of a new-made being describing the
marvels of his incoherent sensations, he subsided, a little dashed,
into listening to their instruction. And the eldest of the blind men
explained to him life and philosophy and religion, how that the world
(meaning their valley) had been first an empty hollow in the rocks,
and then had come first inanimate things without the gift of touch,
and llamas and a few other creatures that had little sense, and then
men, and at last angels, whom one could hear singing and making
fluttering sounds, but whom no one could touch at all, which puzzled
Nunez greatly until he thought of the birds.

He went on to tell Nunez how this time had been divided into the warm
and the cold, which are the blind equivalents of day and night, and
how it was good to sleep in the warm and work during the cold, so that
now, but for his advent, the whole town of the blind would have been
asleep. He said Nunez must have been specially created to learn and
serve the wisdom they had acquired, and that for all his mental
incoherency and stumbling behaviour he must have courage and do his
best to learn, and at that all the people in the door-way murmured
encouragingly. He said the night---for the blind call their day
night---was now far gone, and it behooved everyone to go back to
sleep. He asked Nunez if he knew how to sleep, and Nunez said he did,
but that before sleep he wanted food.  They brought him food, llama's
milk in a bowl and rough salted bread, and led him into a lonely place
to eat out of their hearing, and afterwards to slumber until the chill
of the mountain evening roused them to begin their day again.  But
Nunez slumbered not at all.

Instead, he sat up in the place where they had left him, resting his
limbs and turning the unanticipated circumstances of his arrival over
and over in his mind.

Every now and then he laughed, sometimes with amusement and sometimes
with indignation.

"Unformed mind!" he said. "Got no senses yet! They little know they've
been insulting their Heaven-sent King and master . . . . .

"I see I must bring them to reason.

"Let me think.

"Let me think."

He was still thinking when the sun set.

Nunez had an eye for all beautiful things, and it seemed to him that
the glow upon the snow-fields and glaciers that rose about the valley
on every side was the most beautiful thing he had ever seen. His eyes
went from that inaccessible glory to the village and irrigated fields,
fast sinking into the twilight, and suddenly a wave of emotion took
him, and he thanked God from the bottom of his heart that the power of
sight had been given him.

He heard a voice calling to him from out of the village.

"Yaho there, Bogota! Come hither!"

At that he stood up, smiling. He would show these people once and for
all what sight would do for a man. They would seek him, but not find
him.

"You move not, Bogota," said the voice.

He laughed noiselessly and made two stealthy steps aside from the
path.

"Trample not on the grass, Bogota; that is not allowed."

Nunez had scarcely heard the sound he made himself. He stopped,
amazed.

The owner of the voice came running up the piebald path towards him.

He stepped back into the pathway. "Here I am," he said.

"Why did you not come when I called you?" said the blind man. "Must
you be led like a child? Cannot you hear the path as you walk?"

Nunez laughed. "I can see it," he said.

"There is no such word as SEE," said the blind man, after a
pause. "Cease this folly and follow the sound of my feet."

Nunez followed, a little annoyed.

"My time will come," he said.

"You'll learn," the blind man answered. "There is much to learn in the
world."

"Has no one told you, `In the Country of the Blind the One-Eyed Man is
King?'"

"What is blind?"\ asked the blind man, carelessly, over his shoulder.

Four days passed and the fifth found the King of the Blind still
incognito, as a clumsy and useless stranger among his subjects.

It was, he found, much more difficult to proclaim himself than he had
supposed, and in the meantime, while he meditated his coup d'etat, he
did what he was told and learnt the manners and customs of the Country
of the Blind. He found working and going about at night a particularly
irksome thing, and he decided that that should be the first thing he
would change.

They led a simple, laborious life, these people, with all the elements
of virtue and happiness as these things can be understood by men. They
toiled, but not oppressively; they had food and clothing sufficient
for their needs; they had days and seasons of rest; they made much of
music and singing, and there was love among them and little
children. It was marvellous with what confidence and precision they
went about their ordered world. Everything, you see, had been made to
fit their needs; each of the radiating paths of the valley area had a
constant angle to the others, and was distinguished by a special notch
upon its kerbing; all obstacles and irregularities of path or meadow
had long since been cleared away; all their methods and procedure
arose naturally from their special needs. Their senses had become
marvellously acute; they could hear and judge the slightest gesture of
a man a dozen paces away---could hear the very beating of his
heart. Intonation had long replaced expression with them, and touches
gesture, and their work with hoe and spade and fork was as free and
confident as garden work can be. Their sense of smell was
extraordinarily fine; they could distinguish individual differences as
readily as a dog can, and they went about the tending of llamas, who
lived among the rocks above and came to the wall for food and shelter,
with ease and confidence. It was only when at last Nunez sought to
assert himself that he found how easy and confident their movements
could be.

He rebelled only after he had tried persuasion.

He tried at first on several occasions to tell them of sight. "Look
you here, you people," he said. "There are things you do not
understand in me."

Once or twice one or two of them attended to him; they sat with faces
downcast and ears turned intelligently towards him, and he did his
best to tell them what it was to see. Among his hearers was a girl,
with eyelids less red and sunken than the others, so that one could
almost fancy she was hiding eyes, whom especially he hoped to
persuade. He spoke of the beauties of sight, of watching the
mountains, of the sky and the sunrise, and they heard him with amused
incredulity that presently became condemnatory. They told him there
were indeed no mountains at all, but that the end of the rocks where
the llamas grazed was indeed the end of the world; thence sprang a
cavernous roof of the universe, from which the dew and the avalanches
fell; and when he maintained stoutly the world had neither end nor
roof such as they supposed, they said his thoughts were wicked. So far
as he could describe sky and clouds and stars to them it seemed to
them a hideous void, a terrible blankness in the place of the smooth
roof to things in which they believed---it was an article of faith with
them that the cavern roof was exquisitely smooth to the touch. He saw
that in some manner he shocked them, and gave up that aspect of the
matter altogether, and tried to show them the practical value of
sight. One morning he saw Pedro in the path called Seventeen and
coming towards the central houses, but still too far off for hearing
or scent, and he told them as much. "In a little while," he
prophesied, "Pedro will be here." An old man remarked that Pedro had
no business on path Seventeen, and then, as if in confirmation, that
individual as he drew near turned and went transversely into path Ten,
and so back with nimble paces towards the outer wall. They mocked
Nunez when Pedro did not arrive, and afterwards, when he asked Pedro
questions to clear his character, Pedro denied and outfaced him, and
was afterwards hostile to him.

Then he induced them to let him go a long way up the sloping meadows
towards the wall with one complaisant individual, and to him he
promised to describe all that happened among the houses. He noted
certain goings and comings, but the things that really seemed to
signify to these people happened inside of or behind the windowless
houses---the only things they took note of to test him by---and of those
he could see or tell nothing; and it was after the failure of this
attempt, and the ridicule they could not repress, that he resorted to
force. He thought of seizing a spade and suddenly smiting one or two
of them to earth, and so in fair combat showing the advantage of
eyes. He went so far with that resolution as to seize his spade, and
then he discovered a new thing about himself, and that was that it was
impossible for him to hit a blind man in cold blood.

He hesitated, and found them all aware that he had snatched up the
spade. They stood all alert, with their heads on one side, and bent
ears towards him for what he would do next.

"Put that spade down," said one, and he felt a sort of helpless
horror. He came near obedience.

Then he had thrust one backwards against a house wall, and fled past
him and out of the village.

He went athwart one of their meadows, leaving a track of trampled
grass behind his feet, and presently sat down by the side of one of
their ways. He felt something of the buoyancy that comes to all men in
the beginning of a fight, but more perplexity. He began to realise
that you cannot even fight happily with creatures who stand upon a
different mental basis to yourself. Far away he saw a number of men
carrying spades and sticks come out of the street of houses and
advance in a spreading line along the several paths towards him. They
advanced slowly, speaking frequently to one another, and ever and
again the whole cordon would halt and sniff the air and listen.

The first time they did this Nunez laughed. But afterwards he did not
laugh.

One struck his trail in the meadow grass and came stooping and feeling
his way along it.

For five minutes he watched the slow extension of the cordon, and then
his vague disposition to do something forthwith became frantic. He
stood up, went a pace or so towards the circumferential wall, turned,
and went back a little way.  There they all stood in a crescent, still
and listening.

He also stood still, gripping his spade very tightly in both
hands. Should he charge them?

The pulse in his ears ran into the rhythm of "In the Country of the
Blind the One-Eyed Man is King."

Should he charge them?

He looked back at the high and unclimbable wall behind---unclimbable
because of its smooth plastering, but withal pierced with many little
doors and at the approaching line of seekers. Behind these others were
now coming out of the street of houses.

Should he charge them?

"Bogota!" called one. "Bogota! where are you?"

He gripped his spade still tighter and advanced down the meadows
towards the place of habitations, and directly he moved they converged
upon him. "I'll hit them if they touch me," he swore; "by Heaven, I
will. I'll hit." He called aloud, "Look here, I'm going to do what I
like in this valley! Do you hear? I'm going to do what I like and go
where I like."

They were moving in upon him quickly, groping, yet moving rapidly. It
was like playing blind man's buff with everyone blindfolded except
one. "Get hold of him!" cried one. He found himself in the arc of a
loose curve of pursuers. He felt suddenly he must be active and
resolute.

"You don't understand," he cried, in a voice that was meant to be
great and resolute, and which broke. "You are blind and I can
see. Leave me alone!"

"Bogota! Put down that spade and come off the grass!"

The last order, grotesque in its urban familiarity, produced a gust of
anger.  "I'll hurt you," he said, sobbing with emotion. "By Heaven,
I'll hurt you! Leave me alone!"

He began to run---not knowing clearly where to run. He ran from the
nearest blind man, because it was a horror to hit him. He stopped, and
then made a dash to escape from their closing ranks. He made for where
a gap was wide, and the men on either side, with a quick perception of
the approach of his paces, rushed in on one another. He sprang
forward, and then saw he must be caught, and SWISH!  the spade had
struck. He felt the soft thud of hand and arm, and the man was down
with a yell of pain, and he was through.

Through! And then he was close to the street of houses again, and
blind men, whirling spades and stakes, were running with a reasoned
swiftness hither and thither.

He heard steps behind him just in time, and found a tall man rushing
forward and swiping at the sound of him. He lost his nerve, hurled his
spade a yard wide of this antagonist, and whirled about and fled,
fairly yelling as he dodged another.

He was panic-stricken. He ran furiously to and fro, dodging when there
was no need to dodge, and, in his anxiety to see on every side of him
at once, stumbling. For a moment he was down and they heard his
fall. Far away in the circumferential wall a little doorway looked
like Heaven, and he set off in a wild rush for it. He did not even
look round at his pursuers until it was gained, and he had stumbled
across the bridge, clambered a little way among the rocks, to the
surprise and dismay of a young llama, who went leaping out of sight,
and lay down sobbing for breath.

And so his coup d'etat came to an end.

He stayed outside the wall of the valley of the blind for two nights
and days without food or shelter, and meditated upon the
Unexpected. During these meditations he repeated very frequently and
always with a profounder note of derision the exploded proverb: "In
the Country of the Blind the One-Eyed Man is King." He thought chiefly
of ways of fighting and conquering these people, and it grew clear
that for him no practicable way was possible. He had no weapons, and
now it would be hard to get one.

The canker of civilisation had got to him even in Bogota, and he could
not find it in himself to go down and assassinate a blind man. Of
course, if he did that, he might then dictate terms on the threat of
assassinating them all. But---Sooner or later he must sleep! . . . .

He tried also to find food among the pine trees, to be comfortable
under pine boughs while the frost fell at night, and--- with less
confidence---to catch a llama by artifice in order to try to kill
it---perhaps by hammering it with a stone---and so finally, perhaps, to
eat some of it. But the llamas had a doubt of him and regarded him
with distrustful brown eyes and spat when he drew near.  Fear came on
him the second day and fits of shivering. Finally he crawled down to
the wall of the Country of the Blind and tried to make his terms. He
crawled along by the stream, shouting, until two blind men came out to
the gate and talked to him.

"I was mad," he said. "But I was only newly made."

They said that was better.

He told them he was wiser now, and repented of all he had done.

Then he wept without intention, for he was very weak and ill now, and
they took that as a favourable sign.

They asked him if he still thought he could SEE."

"No," he said. "That was folly. The word means nothing. Less than
nothing!"

They asked him what was overhead.

"About ten times ten the height of a man there is a roof above the
world---of rock---and very, very smooth. So smooth---so beautifully
smooth . . "He burst again into hysterical tears. "Before you ask me
any more, give me some food or I shall die!"

He expected dire punishments, but these blind people were capable of
toleration.  They regarded his rebellion as but one more proof of his
general idiocy and inferiority, and after they had whipped him they
appointed him to do the simplest and heaviest work they had for anyone
to do, and he, seeing no other way of living, did submissively what he
was told.

He was ill for some days and they nursed him kindly. That refined his
submission. But they insisted on his lying in the dark, and that was a
great misery. And blind philosophers came and talked to him of the
wicked levity of his mind, and reproved him so impressively for his
doubts about the lid of rock that covered their cosmic casserole that
he almost doubted whether indeed he was not the victim of
hallucination in not seeing it overhead.

So Nunez became a citizen of the Country of the Blind, and these
people ceased to be a generalised people and became individualities to
him, and familiar to him, while the world beyond the mountains became
more and more remote and unreal. There was Yacob, his master, a kindly
man when not annoyed; there was Pedro, Yacob's nephew; and there was
Medina-sarote, who was the youngest daughter of Yacob. She was little
esteemed in the world of the blind, because she had a clear-cut face
and lacked that satisfying, glossy smoothness that is the blind man's
ideal of feminine beauty, but Nunez thought her beautiful at first,
and presently the most beautiful thing in the whole creation. Her
closed eyelids were not sunken and red after the common way of the
valley, but lay as though they might open again at any moment; and she
had long eyelashes, which were considered a grave disfigurement. And
her voice was weak and did not satisfy the acute hearing of the valley
swains. So that she had no lover.

There came a time when Nunez thought that, could he win her, he would
be resigned to live in the valley for all the rest of his days.

He watched her; he sought opportunities of doing her little services
and presently he found that she observed him. Once at a rest-day
gathering they sat side by side in the dim starlight, and the music
was sweet. His hand came upon hers and he dared to clasp it. Then very
tenderly she returned his pressure. And one day, as they were at their
meal in the darkness, he felt her hand very softly seeking him, and as
it chanced the fire leapt then, and he saw the tenderness of her face.

He sought to speak to her.

He went to her one day when she was sitting in the summer moonlight
spinning.  The light made her a thing of silver and mystery. He sat
down at her feet and told her he loved her, and told her how beautiful
she seemed to him. He had a lover's voice, he spoke with a tender
reverence that came near to awe, and she had never before been touched
by adoration. She made him no definite answer, but it was clear his
words pleased her.

After that he talked to her whenever he could take an opportunity. The
valley became the world for him, and the world beyond the mountains
where men lived by day seemed no more than a fairy tale he would some
day pour into her ears. Very tentatively and timidly he spoke to her
of sight.

Sight seemed to her the most poetical of fancies, and she listened to
his description of the stars and the mountains and her own sweet
white-lit beauty as though it was a guilty indulgence. She did not
believe, she could only half understand, but she was mysteriously
delighted, and it seemed to him that she completely understood.

His love lost its awe and took courage. Presently he was for demanding
her of Yacob and the elders in marriage, but she became fearful and
delayed. And it was one of her elder sisters who first told Yacob that
Medina-sarote and Nunez were in love.

There was from the first very great opposition to the marriage of
Nunez and Medina-sarote; not so much because they valued her as
because they held him as a being apart, an idiot, incompetent thing
below the permissible level of a man.  Her sisters opposed it bitterly
as bringing discredit on them all; and old Yacob, though he had formed
a sort of liking for his clumsy, obedient serf, shook his head and
said the thing could not be. The young men were all angry at the idea
of corrupting the race, and one went so far as to revile and strike
Nunez. He struck back. Then for the first time he found an advantage
in seeing, even by twilight, and after that fight was over no one was
disposed to raise a hand against him. But they still found his
marriage impossible.

Old Yacob had a tenderness for his last little daughter, and was
grieved to have her weep upon his shoulder.

"You see, my dear, he's an idiot. He has delusions; he can't do
anything right."

"I know," wept Medina-sarote. "But he's better than he was. He's
getting better.  And he's strong, dear father, and kind---stronger and
kinder than any other man in the world. And he loves me---and, father,
I love him."

Old Yacob was greatly distressed to find her inconsolable, and,
besides---what made it more distressing---he liked Nunez for many
things. So he went and sat in the windowless council-chamber with the
other elders and watched the trend of the talk, and said, at the
proper time, "He's better than he was. Very likely, some day, we shall
find him as sane as ourselves."

Then afterwards one of the elders, who thought deeply, had an idea. He
was a great doctor among these people, their medicine-man, and he had
a very philosophical and inventive mind, and the idea of curing Nunez
of his peculiarities appealed to him. One day when Yacob was present
he returned to the topic of Nunez. "I have examined Nunez," he said,
"and the case is clearer to me. I think very probably he might be
cured."

"This is what I have always hoped," said old Yacob.

"His brain is affected," said the blind doctor.

The elders murmured assent.

"Now, WHAT affects it?"

"Ah!" said old Yacob.

THIS," said the doctor, answering his own question. "Those queer
things that are called the eyes, and which exist to make an agreeable
depression in the face, are diseased, in the case of Nunez, in such a
way as to affect his brain. They are greatly distended, he has
eyelashes, and his eyelids move, and consequently his brain is in a
state of constant irritation and distraction."

"Yes?" said old Yacob. "Yes?"

"And I think I may say with reasonable certainty that, in order to
cure him complete, all that we need to do is a simple and easy
surgical operation---namely, to remove these irritant bodies."

"And then he will be sane?"

"Then he will be perfectly sane, and a quite admirable citizen."

"Thank Heaven for science!" said old Yacob, and went forth at once to
tell Nunez of his happy hopes.

But Nunez's manner of receiving the good news struck him as being cold
and disappointing.

"One might think," he said, "from the tone you take that you did not
care for my daughter."

It was Medina-sarote who persuaded Nunez to face the blind surgeons.

"YOU do not want me," he said, "to lose my gift of sight?"

She shook her head.

"My world is sight."

Her head drooped lower.

"There are the beautiful things, the beautiful little things---the
flowers, the lichens amidst the rocks, the light and softness on a
piece of fur, the far sky with its drifting dawn of clouds, the
sunsets and the stars. And there is YOU.  For you alone it is good to
have sight, to see your sweet, serene face, your kindly lips, your
dear, beautiful hands folded together. . . . . It is these eyes of
mine you won, these eyes that hold me to you, that these idiots seek.
Instead, I must touch you, hear you, and never see you again. I must
come under that roof of rock and stone and darkness, that horrible
roof under which your imaginations stoop . . . NO; YOU would not have
me do that?"

A disagreeable doubt had arisen in him. He stopped and left the thing
a question.

"I wish," she said, "sometimes---" She paused.

"Yes?" he said, a little apprehensively.

"I wish sometimes---you would not talk like that."

"Like what?"

"I know it's pretty---it's your imagination. I love it, but NOW---"

He felt cold. "NOW?" he said, faintly.

She sat quite still.

"You mean---you think---I should be better, better perhaps---"

He was realising things very swiftly. He felt anger perhaps, anger at
the dull course of fate, but also sympathy for her lack of
understanding---a sympathy near akin to pity.

"DEAR," he said, and he could see by her whiteness how tensely her
spirit pressed against the things she could not say. He put his arms
about her, he kissed her ear, and they sat for a time in silence.

"If I were to consent to this?" he said at last, in a voice that was
very gentle.

She flung her arms about him, weeping wildly. "Oh, if you would," she
sobbed, "if only you would!"

For a week before the operation that was to raise him from his
servitude and inferiority to the level of a blind citizen Nunez knew
nothing of sleep, and all through the warm, sunlit hours, while the
others slumbered happily, he sat brooding or wandered aimlessly,
trying to bring his mind to bear on his dilemma.  He had given his
answer, he had given his consent, and still he was not sure.  And at
last work-time was over, the sun rose in splendour over the golden
crests, and his last day of vision began for him. He had a few minutes
with Medina-sarote before she went apart to sleep.

"To-morrow," he said, "I shall see no more."

"Dear heart!" she answered, and pressed his hands with all her
strength.

"They will hurt you but little," she said; "and you are going through
this pain, you are going through it, dear lover, for ME . . . . Dear,
if a woman's heart and life can do it, I will repay you. My dearest
one, my dearest with the tender voice, I will repay."

He was drenched in pity for himself and her.

He held her in his arms, and pressed his lips to hers and looked on
her sweet face for the last time. "Good-bye!" he whispered to that
dear sight, "good-bye!"

And then in silence he turned away from her.

She could hear his slow retreating footsteps, and something in the
rhythm of them threw her into a passion of weeping.

He walked away.

He had fully meant to go to a lonely place where the meadows were
beautiful with white narcissus, and there remain until the hour of his
sacrifice should come, but as he walked he lifted up his eyes and saw
the morning, the morning like an angel in golden armour, marching down
the steeps . . . .

It seemed to him that before this splendour he and this blind world in
the valley, and his love and all, were no more than a pit of sin.

He did not turn aside as he had meant to do, but went on and passed
through the wall of the circumference and out upon the rocks, and his
eyes were always upon the sunlit ice and snow.

He saw their infinite beauty, and his imagination soared over them to
the things beyond he was now to resign for ever!

He thought of that great free world that he was parted from, the world
that was his own, and he had a vision of those further slopes,
distance beyond distance, with Bogota, a place of multitudinous
stirring beauty, a glory by day, a luminous mystery by night, a place
of palaces and fountains and statues and white houses, lying
beautifully in the middle distance. He thought how for a day or so one
might come down through passes drawing ever nearer and nearer to its
busy streets and ways. He thought of the river journey, day by day,
from great Bogota to the still vaster world beyond, through towns and
villages, forest and desert places, the rushing river day by day,
until its banks receded, and the big steamers came splashing by and
one had reached the sea---the limitless sea, with its thousand islands,
its thousands of islands, and its ships seen dimly far away in their
incessant journeyings round and about that greater world. And there,
unpent by mountains, one saw the sky---the sky, not such a disc as one
saw it here, but an arch of immeasurable blue, a deep of deeps in
which the circling stars were floating . . . .

His eyes began to scrutinise the great curtain of the mountains with a
keener inquiry.

For example; if one went so, up that gully and to that chimney there,
then one might come out high among those stunted pines that ran round
in a sort of shelf and rose still higher and higher as it passed above
the gorge. And then? That talus might be managed. Thence perhaps a
climb might be found to take him up to the precipice that came below
the snow; and if that chimney failed, then another farther to the east
might serve his purpose better. And then? Then one would be out upon
the amber-lit snow there, and half-way up to the crest of those
beautiful desolations. And suppose one had good fortune!

He glanced back at the village, then turned right round and regarded
it with folded arms.

He thought of Medina-sarote, and she had become small and remote.

He turned again towards the mountain wall down which the day had come
to him.

Then very circumspectly he began his climb.

When sunset came he was not longer climbing, but he was far and
high. His clothes were torn, his limbs were bloodstained, he was
bruised in many places, but he lay as if he were at his ease, and
there was a smile on his face.

From where he rested the valley seemed as if it were in a pit and
nearly a mile below. Already it was dim with haze and shadow, though
the mountain summits around him were things of light and fire. The
mountain summits around him were things of light and fire, and the
little things in the rocks near at hand were drenched with light and
beauty, a vein of green mineral piercing the grey, a flash of small
crystal here and there, a minute, minutely-beautiful orange lichen
close beside his face. There were deep, mysterious shadows in the
gorge, blue deepening into purple, and purple into a luminous
darkness, and overhead was the illimitable vastness of the sky. But he
heeded these things no longer, but lay quite still there, smiling as
if he were content now merely to have escaped from the valley of the
Blind, in which he had thought to be King. And the glow of the sunset
passed, and the night came, and still he lay there, under the cold,
clear stars.



\end{document}

